\cleardoublepage
\chapternonum{序言}
增强现实(Augmented Reality)技术是一种将虚拟信息与真实世界巧妙融合的技术,广泛运用了多媒体、三维建模、实时跟踪及注册、智能交互、传感等多种技术手段,将计算机生成的文字、图像、三维模型、音乐、视频等虚拟信息模拟仿真后,应用到真实世界中,两种信息互为补充,从而实现对真实世界的“增强”。

随着传感定位技术的发展和硬件计算能力的提升,作为数字世界与真实世界纽带的AR技术不断有新的进展,基于AR的沉浸式应用大量。然而高可用性的AR技术仍然需要较高的计算能力,相对较低的计算力和能耗约束限制了移动平台在AR中的应用,而边缘计算相对云计算的高实时性使得边缘计算非常适合AR应用场景。

但是目前已有的AR系统,大致可以分为两种,一种是公司推出的商业化AR平台,大公司的平台一般是免费的,而小公司的则收费使用,这些平台普遍是闭源的,无法进行修改。二是学术界验证性的AR平台和一些SLAM系统,这些平台有一些是开源的,能实现AR的功能,但是比较难以部署,也没有各类第三方应用的扩展性。

闭源的AR平台缺乏灵活性,而学术界的AR平台对第三方应用的开发缺乏支撑,实现一个高可靠性和高扩展性的开源AR平台,有利于在其上开发多样的AR应用。比如工业场景下有:设备交互,设备入库登记和位置记录、手动组装,设备维护,过程监控和模拟,质量检查等;商业场景比如设备教程、设备维修辅助、实地生活服务呼叫,会议室登记情况,会议记录等;生活场景如AR游戏、照片墙等。AR平台的应用有着丰富的想象空间。
