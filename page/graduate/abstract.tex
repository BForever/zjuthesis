\cleardoublepage
\chapternonum{摘要}
随着物联网产业的快速发展,越来越多的设备出现在人们身边,据相关机构预测\cite{macgillivray2019worldwide},2025年全世界将安装820亿台物联网设备,人均将拥有十台以上的设备,因此人们将会越来越多地与各类设备进行交互。与此同时,现有的主流人机交互方式如手机APP、智能音箱等均不能实现所见即所得的人机交互,物理空间与数字空间仍然一分为二,用户不能实现直观且自然的交互。

增强现实技术可以很好地融合物理空间与数字空间,增强用户的交互体验。目前的增强现实技术在在对象的定位和渲染方面有了长足的进步,然而在实际的应用中还需要解决很多挑战,在性能方面,为了有效识别动态场景和目标,增强现实技术对设备算力提出较高要求,否则难以满足实时性的需求;在实用性方面,需要实现地图的长期可用性和对象的有效重识别;在交互方面,物联网设备平台已经能很好地集成各类设备,但尚缺乏将其整合于增强现实交互之中为用户提供便捷的交互能力的尝试。

本文提出了VSLink,为基于增强现实的交互方案提供了快速的对象识别定位能力和普适交互能力,结合边缘计算以及物联网设备平台,使增强现实技术能实际应用于实时的人机交互,让用户通过手机的屏幕和摄像头即可与现实世界中的对象进行交互。本文基于已有地图进行对象定位,以减少处理延迟,并利用视觉SLAM和目标检测神经网络的互补特性,让它们分别检测静态和动态目标,动态更新地图以实现地图的长期可用性。本文将VSLAM获取的已知对象区域发送到神经网络,使神经网络可以基于这些可忽略的区域,利用稀疏卷积的方式进行推理加速,以加速对象的重识别。此外,本文还基于物模型和物联网设备平台提供了一个交互界面定制平台,用户可以在其中自定义交互对象/功能/界面布局以定制各类对象在增强现实中可用的交互行为。

本文在一个包含多个交互对象的实验室环境中测试了VSLink。结果表明,该方法能够以30FPS的视频输入进行实时目标定位与识别。本文完成的用户调研和案例验证表明,VSLink能实现灵活的用户定制交互和良好的交互体验。

\noindent\textbf{关键字:}物联网,增强现实,SLAM,稀疏卷积,人机交互

\cleardoublepage
\chapternonum{Abstract}
With the rapid development of the Internet of Things(IoT), people are now surrounded by more and more devices. According to the prediction of relevant institutions\cite{macgillivray2019worldwide}, 82 billion IoT devices will be installed in the whole world in 2025, thus each person will have more than 10 devices on average. Therefore, people now have more and more opportunities to interact with these devices. At the same time, the existing mainstream human-device interaction methods, such as mobile APPs and smart speakers, can not realize WYSIWYG(\textit{What You See Is What You Get}) human-device interaction. Physical space and cyberspace are still separated. Users cannot achieve intuitive and natural interaction. 

Augmented reality technology can well integrate physical space and cyberspace to enhance users' interactive experience. Nowadays, augmented reality technology has made great progress in object localization and rendering. However, challenges need to be solved in practical application. In terms of performance, in order to effectively identify dynamic scenes and targets, augmented reality technology puts forward high requirements for equipment computing power, otherwise, it is difficult to meet the needs of real-time. In terms of practicability, it is necessary to realize the long-term availability of maps and effective re-identification of objects; In terms of interaction, the IoT device platform has been able to integrate many kinds of devices, but there is still a lack of attempt to integrate it into augmented reality interaction to provide users with convenient interaction ability.

In this paper, we propose \textit{VSLink}, which offers fast object identification and localization ability and pervasive interaction for augmented-reality-based interaction solutions. Users can interact with objects in the real world through the screen and camera of their mobile phones.
We locate objects based on existing maps, so as to reduce the processing delay. We make use of the complementary characteristics of visual slam and target detection neural networks to detect static and dynamic targets respectively to dynamically update the map to achieve the long-term availability of the map.
We send the known object regions obtained by VSLAM to the neural network, so that the neural network can accelerate the reasoning by sparse convolution based on these negligible regions, so as to accelerate the object re-recognition.
In addition, we also provide a UI customization platform based on the object model and IoT device platform, in which users can customize the interactive object/function/layout to customize the interactive behavior of various objects available in augmented reality.

We evaluated VSLink in a laboratory environment containing multiple interactive objects. The results showed that this method can locate and recognize real-time targets with 30 FPS video input. Our user research and case verification showed that VSLink can realize flexible user-customized interaction and a well interactive experience.

\noindent\textbf{Keywords:} Internet of Things, Augmented reality, SLAM, Sparse convolution, Human-computer interaction