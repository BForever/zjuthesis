\cleardoublepage
\chapternonum{摘要}
随着物联网的快速发展,人们现在被大量的设备所包围,据相关机构预测,2025年全世界将安装820亿台物联网设备,因此,人们有越来越多的机会与这些设备进行交互。与此同时,人机交互技术也在不断发展,朝着融合物理空间和网络空间的方向发展。手势识别、语音识别、蓝牙、UWB等技术都被用于提高人机交互体验,然而,所见即所得仍然不失为最直观的交互方式,增强现实可以直观地将物理空间和网络空间建立起连接,为用户提供直接的交互。

在本文中,我们提出了VSLink,它为基于增强现实的交互解决方案提供了快速对象识别和普适交互的能力。我们采用两阶段目标识别方法对交互目标进行快速的定位,并利用视觉SLAM和目标检测神经网络的互补特性,让它们分别检测静态和动态目标。我们将SLAM获取的检测先验发送到神经网络,以便神经网络可以基于检测先验,利用稀疏卷积的方式进行推理加速。此外,我们还提供了一个UI定制平台,用户可以在其中自定义交互目标/功能/UI。我们在一个包含多个交互对象的实验室环境中测试了VSLink。结果表明,该方法能够以30FPS的视频输入进行实时目标识别。
\cleardoublepage
\chapternonum{Abstract}
With the rapid development of the Internet of Things(IoT), people are now surrounded by a large number of devices. According to the prediction of relevant institutions, 82 billion IoT devices will be installed in the whole world in 2025. Therefore, people now have more and more opportunities to interact with these devices. At the same time, human-device interaction technology is also evolving, in the direction of fusing physical and cyber space. Gesture recognition, speech recognition, Bluetooth, Ultra Wide Band and other technologies are used to improve the human-device interaction experience. However, \textit{What You See Is What You Get} is still the most intuitive way of interaction. Augmented reality can intuitively establish a connection between physical and cyber space to provide users with direct interaction. 

In this paper, we propose \textit{VSLink}, which offers the ability of fast object identification and pervasive interaction for augmented-reality-based interaction solution.
To improve the processing speed and accuracy, VSLink adopts a two-step object identification method to locate the interaction targets.
In this method, visual SLAM and object detection neural networks detect stable/movable objects separately, and detection prior from SLAM is sent to neural networks which enables sparse-convolution-based inference acceleration.
VSLink also uses a platform where the user could customize the interaction target, function and interface.
We evaluated VSLink in an environment containing multiple objects to interact with. 
The results showed that it achieves a 33\% network inference acceleration on state-of-the-art neural networks, and enables object identification with 30FPS video input.
