\cleardoublepage
\chapternonum{摘要}
随着物联网的快速发展,人们现在被大量的设备所包围。与此同时,人机交互技术也在不断发展,朝着融合物理空间和网络空间的方向发展。在本文中,我们提出了VSLink,它为基于增强现实的交互解决方案提供了快速对象识别和普适交互的能力。采用两阶段目标识别方法对交互目标进行定位。我们利用视觉SLAM和目标检测神经网络的互补特性,让它们分别检测静态和动态目标。SLAM的检测先验被发送到神经网络,神经网络可以实现基于稀疏卷积的推理加速。我们还提供了一个平台,用户可以在其中自定义交互目标/功能/UI。我们在一个包含多个要交互的对象的实验室环境中测试了VSLink。结果表明,该方法能够以30FPS的视频输入进行目标识别。
\cleardoublepage
\chapternonum{Abstract}
With the fast growth of the Internet of Things, people now are surrounded by plenty of devices.
At the same time, human-device interaction technology is also evolving, in the direction of fusing physical and cyber space.
In this paper, we propose \textit{VSLink}, which offers the ability of fast object identification and pervasive interaction for augmented-reality-based interaction solution.
To improve the processing speed and accuracy, VSLink adopts a two-step object identification method to locate the interaction targets.
In this method, visual SLAM and object detection neural networks detect stable/movable objects separately, and detection prior from SLAM is sent to neural networks which enables sparse-convolution-based inference acceleration.
VSLink also uses a platform where the user could customize the interaction target, function and interface.
We evaluated VSLink in an environment containing multiple objects to interact with. 
The results showed that it achieves a 33\% network inference acceleration on state-of-the-art neural networks,, and enables object identification with 30FPS video input.
