
\chapter{备注}
\section{节标题}

我们可以用includegraphics来插入现有的jpg等格式的图片,
如\autoref{fig:zju-logo}所示。

\begin{figure}[htbp]
    \centering
    \includegraphics[width=.3\linewidth]{logo/zju}
    \caption{\label{fig:zju-logo}浙江大学LOGO}
\end{figure}


\subsection{小节标题}


\par 如\autoref{tab:sample}所示,这是一张自动调节列宽的表格。

\begin{table}[htbp]
    \caption{\label{tab:sample}自动调节列宽的表格}
    \begin{tabularx}{\linewidth}{c|X<{\centering}}
        \hline
        第一列 & 第二列 \\ \hline
        xxx & xxx \\ \hline
        xxx & xxx \\ \hline
        xxx & xxx \\ \hline
    \end{tabularx}
\end{table}


\par 如\autoref{equ:sample},这是一个公式

\begin{equation}
    \label{equ:sample}
    A=\overbrace{(a+b+c)+\underbrace{i(d+e+f)}_{\text{虚数}}}^{\text{复数}}
\end{equation}

\chapter{另一章}


\begin{figure}[htbp]
    \centering
    \includegraphics[width=.3\linewidth]{example-image-a}
    \caption{\label{fig:fig-placeholder}图片占位符}
\end{figure}

\chapter{引言}

近年来,物联网(IoT)产业发展迅速。根据[1],2025年将安装820亿台物联网设备。因此,人们有越来越多的机会与这些设备进行交互。物联网设备可能具有人们可以操作的物理接口,例如按钮或开关。一种广泛接受的交互形式是使用应用程序通过另一个智能设备(如手机或笔记本电脑)控制设备[2],[3]。此外,通过语音命令进行交互也变得越来越流行[4],[5]。与上述人机交互形式相比,使用增强现实(AR)技术能够直接显示信息和界面。因此,人们认为它带来的互动打破了物理空间和网络空间之间的界限。

现有的研究试图通过类似AR的设计实现人机交互[6],[7]。Snap-To-It[6]允许用户通过拍摄设备照片并发送到服务器来选择设备。如果照片与任何设备匹配,服务器将返回设备的控制界面并在手机上显示。SnapLink[7]实现了类似的设备控制,而不是使用图像定位方法进行设备识别。然而,他们只能将一台设备识别为一张拍摄的照片,这在速度上是不够的。[8] 研究了边缘辅助的实时移动增强现实技术,实现了高速移动。对于人机交互,[8]缺乏识别设备的能力。[9] -[11]提出的边缘辅助同步定位和映射(SLAM),实现了真正的实时SLAM。这可以用于实现基于AR的人机交互,但集成并非微不足道。
我们认为实现基于AR的人机交互有三个要求:1)视频帧中的设备应被准确识别和定位;2) 处理速度快,用户感觉不到延迟或冻结;3) 交互以用户为导向,以提高体验质量。此外,我们注意到一个有趣的现象,即人们不仅与可连接的设备(即具有基本通信能力的设备)交互,而且与不可连接的对象交互。例如,用户希望保留饲养植物的记录。因此,我们希望将可连接/不可连接的对象都视为交互目标,并且人-设备交互成为人-对象交互。

为了满足上述需求,我们提出了VSLink,它实现了快速对象识别和定制交互的边缘服务。我们设计了一种两步目标识别方法,以保证识别速度和准确性。两步目标识别技术利用视觉SLAM(VSLAM)和目标检测神经网络的互补特性,分别识别稳定/可移动的目标。VSLAM使用视觉特征点描述符来识别存储在地图中的稳定对象。在VSLAM过程之后,它生成一个检测先验,表示存在对象的区域。通过稀疏卷积[12],神经网络的许多计算可以在这样的先验条件下跳过,从而导致显著的推理加速。我们还为用户提供了一个平台,以实现面向用户的交互。使用该平台,用户可以自定义交互目标、功能和界面,无需编码工作。
我们在包含20个对象的真实环境中评估了VSLink。结果表明,该系统支持30fps的视频输入,平均识别率为72.7\%。我们雇佣了10名志愿者通过VSLink实现他们的定制交互,他们确认VSLink的使用,设计过程的平均时间成本在两分钟内。

本文的贡献总结如下:
1.我们提出了VSLink,这是一种基于AR的人机交互方法,融合了物理空间和网络空间。我们提出了一种两步目标识别方法来快速准确地识别目标。我们设计了一个平台来实现面向用户的交互定制。
2.我们实现了VSLink,并在包含多个对象的环境中对其进行了评估。结果表明,在支持30FPS视频输入的情况下,具有很好的速度。
本文的其余部分组织如下。第2节介绍了VSLink的框架。在第3节和第4节中,我们描述了两步对象识别技术和定制交互设计。我们在第5节中介绍了VSLink的部署结果。在第6节中,我们介绍了相关的
作品我们在第8节中总结本文。

\chapter{系统架构}
图1显示了我们提出的VSLink的体系结构。在高层次上,交互涉及三个端,即移动设备端、边缘服务端和对象端。我们可以将交互流程总结如下。首先,用户使用智能手机对周围环境进行视频监控,视频帧通过无线链路发送到边缘服务端。第二,边缘服务识别视频帧中的对象。相应的用户界面(UI)及其位置将发送到智能手机。第三,手机在对象的位置显示UI,用户通过操作UI输入交互命令。最后,该命令由对象/电话/边缘服务根据命令属性进行处理。
为了实现上述流程并实现用户体验的准确性、速度和质量的目标,在边缘服务端我们提出了两个模块,即两步对象识别和对象管理与交互。对象标识模块识别当前帧中的对象,并在低延迟内返回其ID和位置。该模块借鉴VSLAM和目标检测神经网络的功能,实现快速准确的目标识别。具体来说,我们提前构建了环境的对象级SLAM映射。每次启动VSLink时,边缘服务端都会执行一个VSLAM线程。在通过构建的地图定位VSLAM的过程中,我们可以使用视觉特征点描述符匹配来识别稳定的对象。此外,为了处理移动目标,采用了目标检测神经网络。将VSLAM识别结果作为先验,提出了一种基于稀疏卷积的方法,避免了冗余计算。一旦一个物体被神经网络检测到,我们就使用图像检索方法来识别它的ID。
对象管理与交互模块实现了实际的人机交互。如果用户命令是由对象执行的,它会将用户命令发送给对象。例如,命令是打开某个设备。为了转发命令,边缘服务端与这些可连接对象建立连接,并集成它们的API。对于不可连接的对象,VSLink还提供了实现交互的功能,这主要依赖于智能手机的功能。

\chapter{两步物体识别}
\section{动机}
在计算机视觉领域,识别图像/视频中感兴趣的目标已经得到了很好的研究。例如,图像分类[13]、目标检测[14]、图像检索[15]、[16]和图像定位[17]可以实现不同程度的目标识别。在表一中,我们列出了这些方法的特点,但它们都不符合第一节中提到的要求。因此,我们将“目标检测+图像检索”方法与VSLAM相结合,以实现我们的目标识别。我们之所以选择这两种方法,是因为它们在许多方面表现出互补性。我们可以把环境中的物体大致分为两类,稳定的和可移动的。稳定物体往往位于固定位置,例如电视和空调。可移动的通常从一个地方移动到另一个地方。”“对象检测+图像检索”方法(为了简化表示,我们在下面省略图像检索)能够识别所有对象,但由于神经网络计算,速度较慢,而VSLAM识别稳定对象且速度较快。
图2显示了所提出的两步目标识别方法的框架。这种设计接近于实际人脑的工作方式。例如,如果一个人进入卧室,她/他可以立即知道电视的位置,并对自己进行基本定位。然而,要识别手机,此人确实需要注意搜索。直觉是,我们可以利用VSLAM的空间感知快速识别稳定的对象,然后让神经网络处理可移动的对象。同时,神经网络不需要对整个图像进行分析,只需要对其余区域进行分析,大大减少了时间开销。
\section{基于VSLAM的目标识别}
SLAM被认为是实现更真实AR体验的关键技术之一,因为它提供了对环境的理解和跟踪。在VSLink中,我们建议建立一个终身对象级SLAM映射,这有利于SLAM跟踪和对象识别。
1) 工作流:在这里,我们解释了所提出的基于VSLAM的对象识别解决方案的理论。由于经典的VS-LAM[18],[19]技术试图在机器人第一次进入环境时实现精确的映射和定位,我们想知道机器人第二次进入环境时是否能够识别它所看到的物体。
构建的SLAM地图[18]仅包括某些地图点和关键帧,地图重用过程如图3所示。定位过程首先使用单词包(BoW)[20]计算当前帧的表示,然后计算当前帧和关键帧之间的BoW相似性。然后,它选择具有高弓相似性的关键帧作为候选帧。最后,对于每个选定的关键帧,它使用RANSAC[21]算法计算当前帧中的特征点与该关键帧的贴图点之间的2D-3D投影。如果投影误差小于阈值,则定位完成。