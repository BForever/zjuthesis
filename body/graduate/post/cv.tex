\cleardoublepage
\chapternonum{作者简历}
\sectionnonum{教育经历}
	\begin{tabular}{llll}  
		2015.09\textasciitilde2019.06 & 本科 & 浙江大学 & 计算机科学与技术  \\ 
		2019.09\textasciitilde2022.03 & 硕士 & 浙江大学 & 软件工程  \\ 
	\end{tabular}  
\sectionnonum{发表论文}
\begin{enumerate}[label={[\arabic*]}]
    \item Borui Li, \textbf{Hongchang Fan}, Yi Gao, Wei Dong.ThingSpire OS: a WebAssembly-based IoT operating system for cloud-edge integration.MobiSys 2021: 487-488
    \item Wenzhao Zhang, Yuxuan Zhang, \textbf{Hongchang Fan}, Yi Gao, Wei Dong, Jinfeng Wang.TinyEdge: Enabling Rapid Edge System Customization for IoT Applications. SEC 2020: 1-13
    \item Wenzhao Zhang, \textbf{Hongchang Fan}, Yuxuan Zhang, Yi Gao, Wei Dong.Enabling Rapid Edge System Deployment with TinyEdge. SIGCOMM Posters and Demos 2019: 104-106
\end{enumerate}
\sectionnonum{已授权专利}
\begin{enumerate}[label={[\arabic*]}]
    \item 董玮,高艺,张宇轩,张文照,\textbf{范宏昌}. 一种面向边缘计算的消息路由配置方法. 专利号:ZL 202011216229.0. 申请日期:2020 年 11 月 04 日.
\end{enumerate}
\sectionnonum{申请中专利}
\begin{enumerate}[label={[\arabic*]}]
    \item 董玮,高艺,\textbf{范宏昌},周寒. 一种物联网设备交互界面的定制方法. 申请号:202011090471.8. 申请日期:2020年10月13日.
    \item 董玮,高艺,周寒,\textbf{范宏昌}. 一种同步定位和地图构建中物体的快速识别方法. 申请号:202011091518.2. 申请日期:2020年10月13日.
    \item 董玮,高艺,张宇轩,张文照,\textbf{范宏昌}. 一种面向复杂边缘计算的系统校验方法. 申请号:202011169010.X. 申请日期:2020年10月28日.
\end{enumerate}

